%Planificación temporal del proyecto Se recoge la planificación temporal del proyecto o lo que es lo mismo la elaboración del calendario o programa de tiempos. El principal objetivo de esta tarea es recoger de una forma gráfica todas las actividades del proyecto necesarias para producir el resultado final. El destinatario de este documento es el gestor del proyecto, de forma que le facilite su labor de dirección. En el caso de los proyectos de fin de carrera, dado su pequeño tamaño y su reducido número de participantes, se justifica más su presencia por el hecho de que los alumnos aprendan a planificar un calendario, y por la experiencia, en forma de métricas, que estos calendarios pueden aportar para estimaciones de futuros proyectos de mayor entidad. Se aconseja utilizar técnicas gráficas como GANT/PERT [Piattini et al., 1996], [Pressman, 1998] 

%No es infrecuente que en determinados proyectos de fin de carrera vinculados de alguna forma con empresas reales, se solicite un estudio de viabilidad, normalmente centrado en la viabilidad económica del mismo 4 . Precisamente, en relación con la viabilidad económica, la técnica que mayor información puede aportar es el análisid coste/beneficio, que ofrece una valoración de la justificación económica para un proyecto software. 

%En el Cuadro 3 se ofrece el esquema general, basado en el propuesto por Roger S. Pressman [Pressman, 1998], para un documento que recoja el estudio de viabilidad de un proyecto software; el cual puede servir como referencia genérica a adaptar en cada caso.
% Portada
% Lista de cambios
% Tabla de contenidos
% Lista de figuras
% Lista de tablas
% Introducción
% Resumen de gestión y recomendaciones
% Alternativas
% Descripción del sistema
% Análisis coste/beneficio
% Evaluación de riesgo técnico
% Consideraciones legales
% Otros temas específicos del proyecto
% Glosario
