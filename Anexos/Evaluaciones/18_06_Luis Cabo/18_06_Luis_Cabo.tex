\usepackage[T1]{fontenc} % Codificación de las fuentes utilizadas
\usepackage[spanish]{babel} % Español como idioma principal del texto (permite hyphenation de palabras al final de una línea)


\usepackage{graphicx}
\usepackage{url}

\graphicspath{{Figures/}{Diagrams}{Chapters/}}  % Location of the graphics files (set up for graphics to be in PDF format)

\selectlanguage{spanish}

\setcounter{tocdepth}{1}

% Include any extra LaTeX packages required
\usepackage[square, numbers, comma, sort&compress]{natbib}  % Use the "Natbib" style for the references in the Bibliography
\usepackage{verbatim}  % Needed for the "comment" environment to make LaTeX comments
\usepackage{vector}  % Allows "\bvec{}" and "\buvec{}" for "blackboard" style bold vectors in maths
\hypersetup{urlcolor=blue, colorlinks=true}  % Colours hyperlinks in blue, but this can be distracting if there are many links.
\usepackage{hyperref}
% \usepackage[pdfauthor={Diego Martín Arroyo},
%             pdftitle={Diseño e implementación de un sistema de computación distribuida con
% Raspberry Pi, y estudio comparativo del mismo frente a otras soluciones},
%             pdfsubject={Memora del Trabajo de Fin de Grado},
%             pdfproducer={XeLaTeX with hyperref},
%             pdfcreator={XeLaTeX},
%             pdfkeywords={Computación Paralela, Sistema Distribuido, Raspberry}
%             ]{hyperref}
%% ----------------------------------------------------------------

%% --------------------------------------------------------------------------------------------------------------------------------
%http://tex.stackexchange.com/a/85218/76599
\usepackage{fancyvrb}
\usepackage[dvipsnames]{xcolor}

% redefine \VerbatimInput
\RecustomVerbatimCommand{\VerbatimInput}{VerbatimInput}% Inclusión de archivos de texto plano
{fontsize=\footnotesize,
 %
 frame=lines,  % top and bottom rule only
 framesep=2em, % separation between frame and text
 rulecolor=\color{Gray},
 %
 label=\fbox{\color{Black}data.txt},
 labelposition=topline,
 %
 commandchars=\|\(\), % escape character and argument delimiters for
                      % commands within the verbatim
 commentchar=*        % comment character
}

\usepackage{listings} % Requerido para la inserción de código
%Listings command

\usepackage{float}
\newcommand*\lstinputpath[1]{\lstset{inputpath=#1}}
\lstinputpath{Code/}

\newcounter{undefinedreferences}
\setcounter{undefinedreferences}{0}

\newcommand{\citationneeded}[1][None]{\stepcounter{undefinedreferences}\textsuperscript{\color{blue} [Citation needed: #1]}}

\newcommand{\checkreferences}{
	\ifnum\value{undefinedreferences} > 0
	\begin{center}
		\immediate\write18{wget -O Figures/protester.png -nc http://imgs.xkcd.com/comics/wikipedian_protester.png}
		\includegraphics[width=\textwidth]{protester.png}\\
		There are \arabic{undefinedreferences} undefined references
	\end{center}
	\else
	No undefined references. Good!
	\fi
}


%https://github.com/pads-fhs/LaTeX-Template-Thesis/blob/master/lststyles.tex
\lstdefinelanguage{JavaScript}{
  keywords={typeof, new, true, false, catch,%
    function, return, null, catch, switch, var,%
    if, in, while, do, else, case, break},
  ndkeywords={class, export, boolean, throw, implements, import, this},
  sensitive=false,
  comment=[l]{//},
  morecomment=[s]{/*}{*/},
  morestring=[b]',
  morestring=[b]"
}
\newcommand{\lstsetjavascript}{
  \lstset{
		language=JavaScript,
		breaklines=true,
		commentstyle=\textit,
		basicstyle=\ttfamily,
		keywordstyle=\bfseries,
		stringstyle=\ttfamily,
		showstringspaces=false,
		frame=single,
		tabsize=2
  }
}

\lstdefinelanguage{log}{
  keywords={typeof, new, true, false, catch,%
    function, return, null, catch, switch, var,%
    if, in, while, do, else, case, break},
  ndkeywords={class, export, boolean, throw, implements, import, this},
  sensitive=false,
  comment=[l]{//},
  morecomment=[s]{/*}{*/},
  morestring=[b]',
  morestring=[b]"
}
\newcommand{\lstsetlog}{
  \lstset{
		language=log,
		breaklines=true,
		commentstyle=\textit,
		basicstyle=\ttfamily,
		keywordstyle=\bfseries,
		stringstyle=\ttfamily,
		showstringspaces=false,
		frame=single,
		tabsize=2
  }
}

\lstloadlanguages{Java,XML, JavaScript, log}

\newcommand{\javascriptcode}[4]{
	\lstinputlisting[caption=#2,label=#1, firstline=#3, lastline=#4]{#1.json}
}

\newcommand{\logcode}[4]{
	\lstinputlisting[caption=#2,label=#1, firstline=#3, lastline=#4]{#1.log}
}

\usepackage[bottom]{footmisc} %The footnotes go at the bottom of t\usepackage{dtklogos}he page, instead next to the last line.
%Ajustes para Java
% \lstset{
% 	language=java,
%  	frame=single, % Un marco simple alrededor del código
%     basicstyle=\small\ttfamily, % Utilizar fuente true type pequeña
%     keywordstyle=[1]\color{Blue}\bf, % Funciones en negrita y azul
%     keywordstyle=[2]\color{Purple}, % Argumentos en morado
%     keywordstyle=[3]\color{Blue}\underbar, % Funciones personalizadas subrayadas en azul
%     identifierstyle=, % Nada especial acerca de identificadores
%     commentstyle=\usefont{T1}{pcr}{m}{sl}\color{Green}\small, % Los comentarios se renderizan en fuente pequeña verde
%     stringstyle=\color{Purple}, % Cadenas en morado
%     showstringspaces=false, % No se muestran los espacios entre cadenas
%     tabsize=5, % 5 espacios por tabulado
%     %
%     % Put standard Perl functions not included in the default language here
%     %morekeywords={rand},
%     %
%     % Put Perl function parameters here
%     %morekeywords=[2]{on, off, interp},
%     %
%     % Put user defined functions here
%     %morekeywords=[3]{test},\usepackage{dtklogos}
%    	%
%     morecomment=[l][\color{Blue}]{...}, % Line continuation (...) like blue comment
%     numbers=left, % Número de línea a la izquierda
%     firstnumber=1, % Número de línea comienza en 1
%     numberstyle=\tiny\color{Blue}, % Los números de línea son azules y pequeños
%     stepnumber=5, % Los números de línea van de 5 en 5
%     breaklines=true % Salto de línea si el texto no entra. See http://stackoverflow.com/a/1875803
% }

%\usepackage{xltxtra} % XeLaTeX logo. Yep, just that
%http://tex.stackexchange.com/a/73179/76599
\usepackage{metalogo}
\usepackage{dtklogos} %BibTeX logo
\usepackage{xcolor}
\hypersetup{
    colorlinks,
    linkcolor={red!50!black},
    citecolor={blue!50!black},
    urlcolor={blue!80!black}
}

\newcommand{\hmwkTitle}{Evaluación de Deployer, MarcoPolo y distcc} % Assignment title
\newcommand{\hmwkDueDate}{Miércoles,\ 18\ de\ junio\ de\ 2015}
\newcommand{\hmwkClassInstructor}{Rodrigo Santamaría} % Teacher/lecturer
\newcommand{\hmwkAuthorName}{Diego Martín Arroyo} % Your name
\newcommand{\hmwkSubject}{5} % Evaluation subject number

%----------------------------------------------------------------------------------------
%   TITLE PAGE
%----------------------------------------------------------------------------------------
\newcommand{\ordinalindicator}{\hspace{-1.5mm}$\phantom{a}^{\circ}$}
\title{\hmwkTitle}
\author{\textbf{\hmwkAuthorName}}
\date{\hmwkDueDate}

\begin{document}
\maketitle

\tableofcontents


\section{Descripción}

\begin{itemize}
 	\item \textbf{Perfil}: Estudiante del Grado en Ingeniería Informática, con conocimientos de Interacción Persona-Ordenador y Redes.
 	\item No conocía las herramientas previamente.
\end{itemize}


\section{Introducción}

En las fases finales del sistema se deben realizar evaluaciones de las versiones definitivas de los productos \textit{software} creados que interactúen con los usuarios del sistema final. Dichas herramientas en el caso del sistema creado son las diferentes utilidades que componen la herramienta conocida como Deployer, la API de \textbf{MarcoPolo} y las diferentes utilidades de consola que utilizan el protocolo y \texttt{distcc}.

\section{Deployer}

\subsection{Status Monitor}

Se presenta al usuario la utilidad, comenzando en la vista con la utilidad \textbf{Status Monitor}. El sujeto es capaz de comprender de forma independiente cada uno de los componentes visuales, identificando la información que representan. Realiza varias críticas al diseño, como los relojes que muestran el porcentaje de CPU en uso. En su opinión, si bien es un mecanismo efectivo, podría mejorarse utilizando un gráfico de barras.

Se solicita que indique qué información de interés añadiría. Indica que únicamente incluiría el propietario de cada proceso en la vista \textbf{top}.

\subsection{Deployer}

\subsection{Shell}

De la misma forma que con \textbf{Status Monitor}, es capaz de identificar la función de cada elemento, exceptuando la casilla que permite desactivar la ejecución en un nodo, a la cual añadiría una descripción de su funcionalidad.

\subsection{Aspectos generales}

Considera que la integración con el sistema de credenciales presente en la infraestructura (ver \ref{infraestructura}) es una decisión de diseño acertada, si bien incluiría mecanismos más generales, como el acceso a través de la API de Google a fin de facilitar el acceso al sistema.

\section{MarcoPolo}

\subsection{API de MarcoPolo}

Se realiza una demostración del uso de la API de MarcoPolo en Python. La respuesta del sujeto es muy positiva, considerando que su uso es sencillo y las posibilidades que ofrece son numerosas.

El enfoque de la API, que utiliza las mismas convenciones para cualquier lenguaje, así como tipos de datos similares es también valorado positivamente por el sujeto.

Posteriormente se exponen las características técnicas de la implementación del protocolo a petición del usuario. Tras recorrer dichos aspectos, su evaluación es positiva.

En conclusión, considera que el protocolo y su implementación son sencillas de comprender, pues ha conseguido entender de forma básica el principio de funcionamiento en menos de diez minutos.

\subsection{Herramientas de consola}

Se muestran las diferentes herramientas creadas, a saber \texttt{marcodiscover}, {marcoscp}, \texttt{marcoinstallkey} y \texttt{marcosshcommand}. Considera todas ellas de utilidad. No propone ninguna herramienta adicional.

\section{distcc}

Se explica el funcionamiento de distcc, enfocado en su transparencia para el usuario final, fruto de su integración con MarcoPolo. Considera que su uso es muy sencillo y es un buen enfoque para el problema al que propone solución. 


\section{Conclusiones}

A raíz de esta sesión se extraen las siguientes conclusiones y los caminos de actuación:

\begin{itemize}
\item Se realizarán todos los cambios minoritarios propuestos.
\item Se mantiene el sistema de autenticación presente.
\end{itemize}

\end{document}
