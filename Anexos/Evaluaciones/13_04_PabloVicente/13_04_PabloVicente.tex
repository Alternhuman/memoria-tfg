\usepackage[T1]{fontenc} % Codificación de las fuentes utilizadas
\usepackage[spanish]{babel} % Español como idioma principal del texto (permite hyphenation de palabras al final de una línea)


\usepackage{graphicx}
\usepackage{url}

\graphicspath{{Figures/}{Diagrams}{Chapters/}}  % Location of the graphics files (set up for graphics to be in PDF format)

\selectlanguage{spanish}

\setcounter{tocdepth}{1}

% Include any extra LaTeX packages required
\usepackage[square, numbers, comma, sort&compress]{natbib}  % Use the "Natbib" style for the references in the Bibliography
\usepackage{verbatim}  % Needed for the "comment" environment to make LaTeX comments
\usepackage{vector}  % Allows "\bvec{}" and "\buvec{}" for "blackboard" style bold vectors in maths
\hypersetup{urlcolor=blue, colorlinks=true}  % Colours hyperlinks in blue, but this can be distracting if there are many links.
\usepackage{hyperref}
% \usepackage[pdfauthor={Diego Martín Arroyo},
%             pdftitle={Diseño e implementación de un sistema de computación distribuida con
% Raspberry Pi, y estudio comparativo del mismo frente a otras soluciones},
%             pdfsubject={Memora del Trabajo de Fin de Grado},
%             pdfproducer={XeLaTeX with hyperref},
%             pdfcreator={XeLaTeX},
%             pdfkeywords={Computación Paralela, Sistema Distribuido, Raspberry}
%             ]{hyperref}
%% ----------------------------------------------------------------

%% --------------------------------------------------------------------------------------------------------------------------------
%http://tex.stackexchange.com/a/85218/76599
\usepackage{fancyvrb}
\usepackage[dvipsnames]{xcolor}

% redefine \VerbatimInput
\RecustomVerbatimCommand{\VerbatimInput}{VerbatimInput}% Inclusión de archivos de texto plano
{fontsize=\footnotesize,
 %
 frame=lines,  % top and bottom rule only
 framesep=2em, % separation between frame and text
 rulecolor=\color{Gray},
 %
 label=\fbox{\color{Black}data.txt},
 labelposition=topline,
 %
 commandchars=\|\(\), % escape character and argument delimiters for
                      % commands within the verbatim
 commentchar=*        % comment character
}

\usepackage{listings} % Requerido para la inserción de código
%Listings command

\usepackage{float}
\newcommand*\lstinputpath[1]{\lstset{inputpath=#1}}
\lstinputpath{Code/}

\newcounter{undefinedreferences}
\setcounter{undefinedreferences}{0}

\newcommand{\citationneeded}[1][None]{\stepcounter{undefinedreferences}\textsuperscript{\color{blue} [Citation needed: #1]}}

\newcommand{\checkreferences}{
	\ifnum\value{undefinedreferences} > 0
	\begin{center}
		\immediate\write18{wget -O Figures/protester.png -nc http://imgs.xkcd.com/comics/wikipedian_protester.png}
		\includegraphics[width=\textwidth]{protester.png}\\
		There are \arabic{undefinedreferences} undefined references
	\end{center}
	\else
	No undefined references. Good!
	\fi
}


%https://github.com/pads-fhs/LaTeX-Template-Thesis/blob/master/lststyles.tex
\lstdefinelanguage{JavaScript}{
  keywords={typeof, new, true, false, catch,%
    function, return, null, catch, switch, var,%
    if, in, while, do, else, case, break},
  ndkeywords={class, export, boolean, throw, implements, import, this},
  sensitive=false,
  comment=[l]{//},
  morecomment=[s]{/*}{*/},
  morestring=[b]',
  morestring=[b]"
}
\newcommand{\lstsetjavascript}{
  \lstset{
		language=JavaScript,
		breaklines=true,
		commentstyle=\textit,
		basicstyle=\ttfamily,
		keywordstyle=\bfseries,
		stringstyle=\ttfamily,
		showstringspaces=false,
		frame=single,
		tabsize=2
  }
}

\lstdefinelanguage{log}{
  keywords={typeof, new, true, false, catch,%
    function, return, null, catch, switch, var,%
    if, in, while, do, else, case, break},
  ndkeywords={class, export, boolean, throw, implements, import, this},
  sensitive=false,
  comment=[l]{//},
  morecomment=[s]{/*}{*/},
  morestring=[b]',
  morestring=[b]"
}
\newcommand{\lstsetlog}{
  \lstset{
		language=log,
		breaklines=true,
		commentstyle=\textit,
		basicstyle=\ttfamily,
		keywordstyle=\bfseries,
		stringstyle=\ttfamily,
		showstringspaces=false,
		frame=single,
		tabsize=2
  }
}

\lstloadlanguages{Java,XML, JavaScript, log}

\newcommand{\javascriptcode}[4]{
	\lstinputlisting[caption=#2,label=#1, firstline=#3, lastline=#4]{#1.json}
}

\newcommand{\logcode}[4]{
	\lstinputlisting[caption=#2,label=#1, firstline=#3, lastline=#4]{#1.log}
}

\usepackage[bottom]{footmisc} %The footnotes go at the bottom of t\usepackage{dtklogos}he page, instead next to the last line.
%Ajustes para Java
% \lstset{
% 	language=java,
%  	frame=single, % Un marco simple alrededor del código
%     basicstyle=\small\ttfamily, % Utilizar fuente true type pequeña
%     keywordstyle=[1]\color{Blue}\bf, % Funciones en negrita y azul
%     keywordstyle=[2]\color{Purple}, % Argumentos en morado
%     keywordstyle=[3]\color{Blue}\underbar, % Funciones personalizadas subrayadas en azul
%     identifierstyle=, % Nada especial acerca de identificadores
%     commentstyle=\usefont{T1}{pcr}{m}{sl}\color{Green}\small, % Los comentarios se renderizan en fuente pequeña verde
%     stringstyle=\color{Purple}, % Cadenas en morado
%     showstringspaces=false, % No se muestran los espacios entre cadenas
%     tabsize=5, % 5 espacios por tabulado
%     %
%     % Put standard Perl functions not included in the default language here
%     %morekeywords={rand},
%     %
%     % Put Perl function parameters here
%     %morekeywords=[2]{on, off, interp},
%     %
%     % Put user defined functions here
%     %morekeywords=[3]{test},\usepackage{dtklogos}
%    	%
%     morecomment=[l][\color{Blue}]{...}, % Line continuation (...) like blue comment
%     numbers=left, % Número de línea a la izquierda
%     firstnumber=1, % Número de línea comienza en 1
%     numberstyle=\tiny\color{Blue}, % Los números de línea son azules y pequeños
%     stepnumber=5, % Los números de línea van de 5 en 5
%     breaklines=true % Salto de línea si el texto no entra. See http://stackoverflow.com/a/1875803
% }

%\usepackage{xltxtra} % XeLaTeX logo. Yep, just that
%http://tex.stackexchange.com/a/73179/76599
\usepackage{metalogo}
\usepackage{dtklogos} %BibTeX logo
\usepackage{xcolor}
\hypersetup{
    colorlinks,
    linkcolor={red!50!black},
    citecolor={blue!50!black},
    urlcolor={blue!80!black}
}

\newcommand{\hmwkTitle}{Evaluación de las herramientas desarrolladas para MPI} % Assignment title
\newcommand{\hmwkDueDate}{Lunes,\ 13\ de\ April\ de\ 2015} % Due date
\newcommand{\hmwkClassInstructor}{Rodrigo Santamaría} % Teacher/lecturer
\newcommand{\hmwkAuthorName}{Diego Martín Arroyo} % Your name
\newcommand{\hmwkSubject}{4} % Evaluation subject number

%----------------------------------------------------------------------------------------
%   TITLE PAGE
%----------------------------------------------------------------------------------------
\newcommand{\ordinalindicator}{\hspace{-1.5mm}$\phantom{a}^{\circ}$}
\title{\hmwkTitle}
\author{\textbf{\hmwkAuthorName}}
\date{\hmwkDueDate}

\begin{document}
\maketitle

\tableofcontents


\section{Descripción}

\begin{itemize}
 	\item \textbf{Perfil}: Estudiante del Grado Ingeniería Informática.
 	\item El estudiante voluntario cursa la asignatura Arquitectura de Computadores y ya ha realizado la práctica de MPI obligatoria.
\end{itemize}


\section{Introducción}

En el programa de estudios de la titulación Grado en Ingeniería Informática de la Universidad de Salamanca se incluye la adquisición de conocimientos básicos de la división de tareas entre diferentes procesos situados en un único equipo o en un conjunto de computadores físicamente independientes, así como los fundamentos básicos que posibilitan la comunicación entre procesos. Como herramienta para la adquisición de dichos conocimientos se plantea la realización de ejercicios prácticos con la interfaz MPI (\textit{Message Passing Interface}) en el marco de la asignatura \textbf{Arquitectura de Computadores}.

Con el objetivo de facilitar el desarrollo de estas prácticas se han creado una serie de herramientas que complementan al sistema, y que en el presente documento sometemos a evaluación.



\section{Evaluación de la herramienta \texttt{marcodiscover.py}}

Se presentan al usuario las soluciones propuestas a los problemas típicos a los que los estudiantes se enfrentan en la asignatura Arquitectura de Computadores, a saber:

\begin{itemize}

\item Detección de cada nodo: Con la utilidad \texttt{marcodiscover.py} el usuario puede conocer las direcciones de todos los nodos sin tener que acceder físicamente a cada uno de ellos. Dado que para poder utilizar un programa en MPI en varios nodos es necesario conocer la IP de cada uno previamente, valora positivamente dicha herramienta.

\item Lanzamiento de programas: La integración de \texttt{marcodiscover.py} en el lanzamiento de un programa para varios nodos es muy sencillo, dado que únicamente es necesario volcar la salida de marcodiscover.py en un fichero, o utilizar un pseudofichero en la shell:

\texttt{marcodiscover.py >\textgreater hosts.txt\\
mpirun -np 20 -hostfile hosts.txt ./ejecutable}

o 

\texttt{mpirun -np 20 -hostfile <(marcodiscover.py) ./ejecutable}

\end{itemize}


\section{Instalación de claves públicas}

La comunicación entre nodos se realiza mediante sesiones remotas SSH, por lo que utilizar una clave RSA evita tener que indicar el usuario y contraseña en cada nodo. Para ello se ha creado el script marcoinstallkey.sh que utiliza internamente la utilidad \texttt{ssh-copy-id}.


El usuario valora positivamente esta utilidad.

\section{Despliegue de los ejecutables}

Uno de los problemas al que los estudiantes se enfrentan es la copia de los ejecutables en cada nodo. Para ello se está creando una herramienta de despliegue que permitirá realizar dicho despliegue. Debido a que aún no está completa se ha preguntado al sujeto qué valoraría en esta herramienta:

Una forma de conocer el árbol de directorios del propio nodo, para seleccionar el directorio de instalación.
La integración con la herramienta de estadísticas y el resto de componentes.

\end{document}
