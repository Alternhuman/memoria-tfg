\chapter{MarcoPolo}

\section{Elicitación de requisitos}

MarcoPolo surge con el objetivo de facilitar la intercomunicación entre los diferentes componentes del sistema distribuido a crear. Como se detalla en la memoria del Trabajo %(ver \ref{})
, los únicos métodos factibles para conocer la localización en la red en la que se integra el sistema es a través del acceso físico a dicha máquina o mediante técnicas de \textit{broadcasting}, muy poco eficientes y en ocasiones poco eficaces. Estas soluciones no permiten extender el proceso de descubrimiento para facilitar el descubrimiento de servicios o conocer información sobre el sistema, por lo que se desestiman de forma inmediata. Una alternativa viable es Avahi, si bien no permite la coexistencia de varios conjuntos de nodos (mallas) independientes en la misma subred. Es por ello que se apuesta por el desarrolo de un sistema propio que permita realizar el proceso de descubrimiento de servicios y nodos.

La definición de requisitos comienza con el planteamiento inicial ya presentado, si bien durante esta fase y posteriores los requisitos planteados se refinarán para incluir mayores prestaciones y se añadirán nuevos. El proceso de elicitación se ha realizado junto con el resto de \textit{stakeholders} del sistema apostando por métodos como las entrevistas y las evaluaciones.

\subsection{Descubrimiento de nodos}

\subsection{Descubrimiento de servicios}

\subsection{Exploración de un nodo}

\subsection{Especificación de criterios}

\subsection{Gestión de errores}

\subsection{Delegación}

\subsection{Reparto de la funcionalidad}

\subsection{Integración}

\subsection{Documentación}

\subsection{Python}

\subsection{Interconexión}

\subsection{Test unitarios}

\subsection{Configuración mediante parámetros}

\subsection{Servicios: definición}

\subsection{JSON}

\subsection{UTF-8}

\subsection{Multicast}

\subsection{\textit{Daemon}}

\subsection{Ficheros para servicios}

\subsection{Servicios persistentes y volátiles}

\subsection{Interconexión con productos software}

\subsection{Integración con servicios del sistema}

\subsection{Separación del protocolo y la implementación}

\subsection{Comunicación sin conexión}



\subsection{Requisitos funcionales}

\subsection{Requisitos no funcionales}

\subsection{Otros requisitos}

\section{Casos de uso}

\subsection{Casos de uso externos}

\subsubsection{Búsqueda de todos los nodos en el sistema}

Utilizando el módulo Marco un actor es capaz de buscar los nodos existentes en la red. Esta acción es llevada a cabo por los usuarios activos, servicios del sistema o paquetes software que utilicen MarcoPolo.


\subsection{Casos de uso internos}

\section{Actores}

Se identifican los siguientes actores:

\begin{itemize}
\item Paquetes \textit{software} que utilizan MarcoPolo.
Se conectan a MarcoPolo mediante el uso de \textit{bindings}.

\item Nodos.
Difunden peticiones utilizando el protocolo MarcoPolo, que serán recibidas por la instancia local, y en caso necesario, se emitirá una respuesta.

\item Usuarios \textit{pasivos}

Son usuarios que utilizan MarcoPolo de forma indirecta, por ejemplo, en el proceso de inicio de sesión. No realizan una interacción directa con el \textit{software} pero es utilizado.

\item Usuarios activos

Son conscientes del uso que están haciendo de MarcoPolo, y realizan interacciones de forma directa con la implementación del protocolo, o bien lo utilizan a través de \textit{bindings}.

\item Servicios del sistema

Servicios como PAM, Tomcat, Hadoop o MarcoManager aprovechan MarcoPolo para facilitar su uso en el sistema.
\section{Escenarios}
