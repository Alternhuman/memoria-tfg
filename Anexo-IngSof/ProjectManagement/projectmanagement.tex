\documentclass{article}

%\usepackage[spanish]{babel}
%\selectlanguage{spanish}
%\usepackage{hyperref}

% Margins
%\topmargin=-0.45in
%\evensidemargin=0in
%\oddsidemargin=0in
%\textwidth=6.5in
%\textheight=9.0in
%\headsep=0.25in 

\linespread{1.1} % Line spacing


%\setlength\parindent{0pt} % Removes all indentation from paragraphs
%\usepackage[utf8]{inputenc}
%\usepackage[T1]{fontenc}
%\usepackage{graphicx}
%\usepackage{float}

\begin{document}
\title{Gestión de proyectos}
\author{Diego Martín}
\maketitle

Uno de los aspectos fundamentales a la hora de realizar un proyecto de desarrollo con una serie de restricciones temporales es la planificación del mismo. El sistema desarrollado presenta además una serie de dificultades adicionales que deben ser tomadas en cuenta a la hora de establecer una metodología de desarrollo.

\section{}


\end{document}