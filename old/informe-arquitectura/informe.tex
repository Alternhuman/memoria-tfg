% !TEX encoding = UTF-8 Unicode
%%%%%%%%%%%%%%%%%%%%%%%%%%%%%%%%%%%%%%%%%
% Informe sobre la práctica "100 metros lisos con REST"
% Perteneciente a la asignatura "Sistemas Distribuidos"
% 
%
% Licencia:
% CC BY-NC-SA 3.0 (http://creativecommons.org/licenses/by-nc-sa/3.0/)
%
%%%%%%%%%%%%%%%%%%%%%%%%%%%%%%%%%%%%%%%%%

%----------------------------------------------------------------------------------------
%	PACKAGES AND DOCUMENT CONFIGURATIONS
%----------------------------------------------------------------------------------------

\documentclass{article}

\usepackage[T1]{fontenc} % Codificación de las fuentes utilizadas
\usepackage[spanish]{babel} % Español como idioma principal del texto (permite hyphenation de palabras al final de una línea)
\usepackage[pdfauthor={Diego Martín Arroyo},
            pdftitle={Arquitectura del sistema},
            pdfsubject={Informe sobre la arquitectura inicial del sistema},
            pdfproducer={XeLaTeX with hyperref},
            pdfcreator={XeLaTeX},
            pdfkeywords={Arquitectura, Sistema, Raspberry}
            ]{hyperref}

\usepackage{graphicx}
\usepackage{url}

% Márgenes
\topmargin=-0.45in
\evensidemargin=0in
\oddsidemargin=0in
\textwidth=6.5in
\textheight=8.5in
\headsep=0.25in

\graphicspath{{./img/}}
\selectlanguage{spanish}

\title{Requisitos básicos del sistema Raspberry Pi a construir}
\author{Diego Martín Arroyo}
\date{\today} % Fecha

\newcounter{undefinedreferences}
%\newcommand{\contador}{
%	\stepcounter{undefinedreferences}
%}

\begin{document}

\maketitle

\begin{abstract}
Resumen de los aspectos a considerar en el desarrollo del Trabajo de Fin de Grado
\end{abstract}

\section{Histórico de cambios}



\section{Introducción}

El presente documento recoge los diferentes aspectos a valorar en el modelado del sistema a construir teniendo en cuenta diferentes criterios que se detallarán más adelante.

Este documento no tiene como objetivo establecer de forma definitiva la arquitectura y aspectos a considerar del sistema, si no que su principal uso es la elicitación de diferentes decisiones de diseño y el uso del mismo para exponer las ideas reflejadas a terceros como tutores o colaboradores.

La técnica utilizada en la versión actual se basa remotamente en \cite{Toro99arequirements}, si bien la influencia de dicha publicación será mayor en posteriores versiones.

\subsection{Objetivos del proyecto}

Este proyecto cuenta con varios objetivos muy diferentes entre sí, que se agrupan en tres categorías:
\begin{itemize}
  \item Arquitectura subyacente\\
  Definición de los componentes hardware a utilizar en el sistema, interconexión de los mismos, soluciones de alimentación eléctrica, almacenamiento\dots.
  \item Servicios a proveer\\
  \item Componente didáctico\\
\end{itemize}


\section{Definiciones\protect\footnote{Esta parte abarca únicamente el componente didáctico del sistema debido a que no se cuenta con ningún tipo de trasfondo para el resto de partes del dominio del problema.}}

\subsection{Definición del dominio del problema}

El sistema se ubica en una Facultad universitaria con aproximadamente 600 alumnos \arabic{undefinedreferences} con varias asignaturas en las que se imparten áreas de conocimiento relacionados con la Computación Distribuida, en particular las asignaturas \textbf{Arquitectura de Computadores} y \textbf{Sistemas Distribuidos} \cite{DIA15GuiaAcademica}.

\subsection{Modelado del sistema actual}

La Facultad cuenta con varias aulas y laboratorios informáticos donde los alumnos disponen de la intraestructura necesaria para realizar los ejercicios y prácticas asignadas. Dichos espacios permiten utilizar cualquier equipo como nodos, pues pertenecen a la misma red, siendo incluso factible la comunicación directa entre equipos situados en diferentes aulas o edificios. La conexión es relativamente rápida, contando con un cableado capaz de soportar teóricamente transferencias de hasta 100Mb/s \textit{full-duplex}. La gestión de usuarios se realiza mediante un protocolo LDAP (\textit{Lightweight Directory Access Protocol}) \cite{RFC4516-comment}, contando con un sistema de ficheros centralizado que permite acceder a la información de un usuario desde cualquier equipo, facilitando las tareas de replicación menos sofisticadas.

La mayoría de las prácticas son programadas en el lenguaje \textbf{Java}, que es ya conocido por la totalidad de los estudiantes gracias a asignaturas previamente cursadas \stepcounter{undefinedreferences} y que facilita el despliegue y la compatibilidad entre diferentes equipos de trabajo sustancialmente.

\paragraph{Problemas conocidos}

Estos son varios de los problemas identificados en los diferentes usuarios de la infraestructura:

\begin{itemize}
  \item Cada pareja de alumnos necesita tres estaciones de trabajo para poder realizar algunos de losejercicios propuestos.
  \item El servidor LDAP constituye un ``cuello de botella'', pues todos los alumnos acceden a él de forma intensiva, provocando caídas en el mismo.
  \item Las técnicas de programación utilizadas hasta la fecha tienen un rendimiento bajo y son en ocasiones relativamente complejas.
\end{itemize}

\subsection{Identificación de usuarios participantes}

\begin{itemize}

  \item Estudiantes de tercero y cuarto curso del Grado en Ingeniería Informática
  \item Doctentes de las asignaturas Arquitectura de Computadores y Sistemas Distribuidos
  \item Administradores del sistema
\end{itemize}

\section{Identificación de las necesidades de cada parte}
\subsection{Necesidades de los alumnos}

\begin{itemize}
  \item Entorno de trabajo útil y sencillo.
  \item Posibilidad de observar los resultados de las ejecuciones de forma sencilla.
\end{itemize}

\subsection{Docentes}

\begin{itemize}
  \item Entorno versátil sobre el cual puedan llevarse a cabo \textbf{todas} las prácticas y ejercicios propuestos, aportando si es posible algún tipo de ventaja sobre el sistema en uso.
\end{itemize}
\subsection{Administrador}
\begin{itemize}
  \item Sistema integrable en la intraestructura actual cuyo mantenimiento sea sencillo.
\end{itemize}

\section{Propuestas para la búsqueda de necesidades}

\begin{itemize}
  \item Encuestas o entrevistas a todas las partes.
  \item Evaluación de la experiencia de uso en las diferentes etapas de desarrollo del sistema.
\end{itemize}

\section{Identificación de requisitos}

\subsection{Requisitos de almacenamiento de la información}

\begin{itemize}
  \item Gestión de usuarios (credenciales de autenticación)
  \item Gestión de los datos de cada usuario
  \item \textit{Logs} del sistema
\end{itemize}

\subsection{Identificación de requisitos funcionales}


\subsection{Identificación de actores}

\subsection{Identificación de requisitos no funcionales}

\begin{itemize}
  \item El \textit{software} debe ser mantenible y robusto\footnote{Siendo dicha robustez garantizada mediante el uso de \textit{software} utilizado por una base de usuarios significativa, una arquitectura conocida, pruebas realizadas sobre él o un equipo de desarrollo en activo, entre otras}.
  \item Reducción de los costes de desarrollo.
  \item Definición de los protocolos de comunicación.
  \item Definición de los protocolos de seguridad y confidencialidad.
  \item Definición de la interacción con el usuario.
  \item Integridad del sistema y fiabilidad (\textit{uptime}, recuperación frente a fallos).
  \item Productos a crear.
  \item Compatibilidad con las prácticas y ejercicios.
\end{itemize}

\bibliographystyle{ieeetr}

\bibliography{informe}

\end{document}