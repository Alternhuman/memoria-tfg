\addtotoc{Resumen}  % Add the "Abstract" page entry to the Contents
\abstract{
\addtocontents{toc}{\vspace{1em}}  % Add a gap in the Contents, for aesthetics

El uso de un sistema distribuido como alternativa económica a un equipo de altas prestaciones es una práctica aprovechada desde hace décadas todo tipo de entornos. El auge actual de sistemas embebidos con gran potencia de cálculo y bajo coste en ha motivado su utilización para la construcción de este tipo de sistemas de computación. Sin embargo, la mayoría de las soluciones responden a una necesidad específica. En el presente documento y los anexos que lo acompañan se plantea y desarrolla la creación de un sistema distribuido compuesto por placas Raspberry Pi que es aprovechable por un gran rango de usuarios con diferentes necesidades (investigadores, desarrolladores, estudiantes\dots), destacando su papel como herramienta didáctica. El resultado final consiste en un sistema escalable y autoconfigurado que incluye un amplio abanico de herramientas de desarrollo de algoritmos distribuidos.

Todas las herramientas creadas cumplen los requisitos de transparencia encontrados en cualquier sistema distribuido. Sumado a su carácter autoconfigurado, el sistema es una solución autosuficiente que gracias al uso de diferentes mecanismos de optimización, hace de este una herramienta de coste mínimo una herramienta con una gran potencia de cálculo.

La construcción del sistema ha implicado el desarrollo de protocolos, utilidades y aplicaciones en diferentes capas, desde servicios provistos por el sistema operativo, pasando por utilidades consumibles por aplicaciones hasta aplicaciones que interactúan con usuarios finales. Todas estas adiciones son transparentes al usuario y altamente integrables con \textit{software} preexistente.

El proceso de desarrollo es precedido por una serie de etapas de evaluación y de un estudio de las diferentes alternativas valoradas. Otro de los frutos del desarrollo del sistema es, por tanto, un estudio comparativo de diferentes propuestas de solución a un problema concreto y de interés en diversas áreas de las Ciencias de la Computación. Otro de los principales componentes del desarrollo es el carácter didáctico del mismo, dado que uno de los objetivos definidos es su utilización en un contexto universitario, facilitando el análisis, desarrollo y comprensión del paradigma de computación distribuida.

\textbf{Palabras clave}: zeroconf, sistema distribuido, plataforma orientada a servicios, enseñanza, marcopolo, raspberry pi, python.
}

\clearpage

\addtotoc{\textit{Abstract}}  % Add the "Abstract" page entry to the Contents
\englishabstract{
\addtocontents{toc}{\vspace{1em}}  % Add a gap in the Contents, for aesthetics

The usage of distributed systems as an economic alternative to a high performace computers is a practice applied in all sorts of environments for decades. The ongoing rise of embedded systems with great computational power and low cost has triggered their usage for that sort of computational systems. However, the majority of proposals are focused on a specific need. This document and the attached appendices, annexes and technical documents cover the proposal and development of a distributed system made of Raspberry Pi boards which is usable by a broader range of users with different needs (researchers, developers, students\dots), featuring its role as a didactic tool. The final result is a scalable and autoconfigured system with a big set of distributed algorithms development tools.

All the development tools fulfill the typical transparency requirements found in any distributed system. Added to its autoconfigured nature, the system is a self-sustaining solution which, thanks to several optimization mechanisms, offers a big computing power with very little cost.

The development of the system involves the creation of protocolos, utilities and applications in a layered architecture, from operating system services to internal utilities and user applications. All the internals are transparent to the user and highly compatible with preexisting software.

The development process is preceded by a series of evaluation phases and an analysis on different alternatives. One of the deliverables of the project is therefore a comparative study on several proposals to a known problem in several areas of Computer Science.

Another key components of the development process is the didactic properties of it. One of the main goals is the usage of the product in a college environment, easing the analysis, development and understanding of the Distributed Computation paradigm fundamentals.

\textbf{Keywordks}: zeroconf, distributed system, service-oriented architecture, teaching, marcopolo, raspberry pi, python.

}

\clearpage  % Abstract ended, start a new page