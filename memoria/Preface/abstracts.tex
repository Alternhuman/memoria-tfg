% The Abstract Page
\addtotoc{Resumen}  % Add the "Abstract" page entry to the Contents
\abstract{
\addtocontents{toc}{\vspace{1em}}  % Add a gap in the Contents, for aesthetics

El uso de un sistema distribuido como alternativa a un equipo de altas prestaciones es una práctica aprovechada desde hace décadas en todo tipo de entornos. El auge de sistemas embebidos con gran potencia de cálculo y bajo coste en la actualidad ha motivado su utilización para la construcción de este tipo de sistemas de computación. Sin embargo, la mayoría de las soluciones responden a una necesidad específica. En el presente documento y los anexos que lo acompañan se plantea y desarrolla la creación de un sistema distribuido compuesto por placas Raspberry Pi que es aprovechable por un gran rango de usuarios con diferentes necesidades (investigadores, desarrolladores\dots), destacando su papel como herramienta didáctica. El resultado final consiste en un sistema escalable y autoconfigurado que incluye un amplio abanico de herramientas de desarrollo.

El desarrollo del sistema ha implicado la construcción de protocolos, utilidades y aplicaciones en diferentes capas, desde servicios provistos por el sistema operativo, pasando por utilidades consumibles por aplicaciones hasta aplicaciones que interactúan con usuarios finales. Todas estas adiciones son transparentes al usuario y altamente integrables con \textit{software} preexistente.

El proceso de desarrollo es precedido por una serie de etapas de evaluación y de un estudio de las diferentes alternativas valoradas antes de proceder a la construcción del sistema.

}

\clearpage  % Abstract ended, start a new page

% The Abstract Page
\addtotoc{\textit{Abstract}}  % Add the "Abstract" page entry to the Contents
\englishabstract{
\addtocontents{toc}{\vspace{1em}}  % Add a gap in the Contents, for aesthetics

English abstract. \citationneeded[TODO]

}

\clearpage  % Abstract ended, start a new page