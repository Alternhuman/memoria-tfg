\lhead{\emph{Introducción}}
\chapter{Introducción}

Los límites físicos de los que adolecen los computadores en la actualidad\citationneeded hacen de la computación distribuida un recurso para incrementar de forma sencilla y económica el rendimiento total de un sistema.%, rendimiento que aumenta significativamente en problemas \textit{ridículamente paralelos}\citationneeded.

Sin embargo, estas ganancias conllevan una serie de inconvenientes entre los que figura el aumento de la complejidad del sistema. En general, un sistema distribuido requiere un conjunto de entidades independientes, más difíciles de configurar y mantener que una única entidad. Además, aparecen nuevos problemas: comunicación, integridad y sincronización, dificultad en el desarrollo y depuración de aplicaciones, etcétera.

En el apartado didáctico, el estudio del paradigma distribuido suele requerir un gran esfuerzo por parte de los estudiantes, en particular a la hora de comprender los fundamentos básicos de cualquier aplicación distribuida.

La presente memoria recoge el proceso de evaluación de diferentes alternativas para la creación de un sistema distribuido que satisfaga un conjunto de necesidades preestablecidas, así como las diferentes etapas de diseño y desarrollo de un sistema formado por dispositivos Raspberry Pi como propuesta de solución y la creación de un conjunto de protocolos, herramientas y programas para la utilización del mismo como utilidad de investigación en el campo de la computación distribuida y como herramienta didáctica para disciplinas relacionadas con dicho área.

El sistema se compone de un conjunto de dispositivos físicos compuesto por los nodos de computación y una serie de módulos accesorios, así como los diferentes mecanismos de alimentación y refrigeración, un conjunto de herramientas \textit{software} que permiten la coordinación y comunicación entre los diferentes procesos y una serie de herramientas que facilitan el trabajo con el sistema.

Además, se incluyen las definiciones de los diferentes conceptos teóricos necesarios para la creación del sistema, así como las diferentes etapas de aprendizaje, evaluación de alternativas y diferentes procesos de evaluación llevados a cabo durante las diferentes etapas desarrollo, así como las metodologías de trabajo utilizadas, sin olvidar la documentación de todas las herramientas creadas.