\lhead{\emph{Introducción}}
\chapter{Introducción}

\begin{cabstract}
En el que se resumen los objetivos a cumplir durante el desarrollo del proyecto y los resultados finales.
\end{cabstract}

Los límites físicos de los que adolecen los computadores en la actualidad \cite{seth:physical} hacen de la computación distribuida un recurso para incrementar de forma sencilla y económica el rendimiento total de un sistema. Sumado a estos beneficios, el auge de sistemas como teléfonos inteligentes o aplicaciones web ha potenciado el auge de diferentes paradigmas distribuidos en los últimos años. 

Sin embargo, estas ganancias conllevan una serie de inconvenientes entre los que figura el aumento de la complejidad del sistema. En general, un sistema distribuido requiere un conjunto de entidades independientes, más difíciles de configurar y mantener que una única entidad. Además, aparecen nuevos problemas: comunicación, integridad y sincronización, dificultad en el desarrollo y depuración de aplicaciones, etcétera.

En el apartado didáctico, el estudio del paradigma distribuido suele requerir un gran esfuerzo por parte de los estudiantes, en particular a la hora de comprender los fundamentos básicos de cualquier aplicación distribuida así como a la hora de realizar tareas de análisis y depuración.

Otro de los problemas que acompañan a la utilización de este tipo de sistemas reside en las comunicaciones entre nodos. Ejemplos de este tipo de problemas son la identificación de los diferentes componentes y sus propiedades o los canales de comunicación a establecer y su eficiencia y fiabilidad.

A dichas dificultades técnicas se suman otras de carácter económico o logístico. Generalmente el coste de estos equipos es elevado (si bien la relación coste/beneficio es muy atractiva) y presentan una serie de requisitos de espacio, instalación y mantenimiento difíciles de satisfacer por algunas organizaciones.

La presente memoria recoge el proceso de evaluación de diferentes alternativas para la creación de un sistema distribuido que define entre sus objetivos funcionales el carácter multipropósito, la autoconfiguración de los diferentes componentes y la reducción del coste total, utilizando equipos ya existentes en la organización donde se integrará o adquiriendo componentes económicos. Posteriormente se describirán las diferentes etapas de diseño y desarrollo de una propuesta de solución al problema: un sistema formado por dispositivos Raspberry Pi y un conjunto de protocolos, herramientas y servicios para la utilización del mismo como plataforma de investigación en el campo de la computación distribuida y como herramienta didáctica para disciplinas relacionadas con dicho área.

El sistema se compone de un conjunto de dispositivos físicos compuesto por los nodos de computación y una serie de módulos accesorios, así como los diferentes mecanismos de alimentación y refrigeración, un conjunto de paquetes \textit{software} que permiten la coordinación y comunicación entre los diferentes procesos y una serie de herramientas que facilitan el trabajo con el sistema.

Además, se incluyen las definiciones de los diferentes conceptos teóricos necesarios para la creación del sistema, así como las diferentes etapas de aprendizaje, análisis de alternativas y diferentes procesos de evaluación llevados a cabo durante las diferentes etapas desarrollo, así como las metodologías de trabajo utilizadas, sin olvidar la documentación de todas las herramientas creadas.

El producto final creado prueba que la utilización de sistemas embebidos de bajo coste, si bien limitados en rendimiento, constituyen una alternativa económica frente a soluciones como el empleo de equipos de escritorio o computadores diseñados de forma específica para este propósito.