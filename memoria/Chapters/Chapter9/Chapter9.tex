\chapter{Herramientas de terceros}
\lhead{\emph{Herramientas de terceros}}

\section{Herramientas utilizadas para la creación del sistema}



\subsection{Lenguajes de programación}

\subsubsection{Python}
Python es un lenguaje de programación interpretado de propósito general que prioriza la legibilidad del  código y la rapidez de desarrollo, manteniendo estas propiedades en proyectos de cualquier escala. Python soporta diferentes paradigmas de programación, entre ellos la orientación a objetos, programación imperativa y la programación funcional. Automatiza la gestión de memoria y utiliza un sistema de tipado dinámico rígido\citationneeded.

\subsubsection{Lenguajes}

Python se ha elegido como lenguaje principal en el desarrollo de herramientas para el sistema en detrimento de otras opciones:

\begin{landscape}
\begin{table}[H]
\begin{tabular}{|p{1.6cm}|p{4.3cm}|p{5.4cm}|p{5.5cm}|p{4.8cm}|}
\hline
Nombre & Características notables & Ventajas & Inconvenientes & Inclusión en el sistema\\ \hline

\textbf{Python} & Orientación a objetos, portable & Portable, buen rendimiento & Necesidad de un intérprete & Se incluye en los componentes de alto nivel del sistema.\\ \hline

\textbf{C} & Imperativo, acceso a características de muy bajo nivel de forma sencilla & El desarrollo en el lenguaje suele ser más complejo que en otros lenguajes de más alto nivel & Se incluye en componentes que trabajan con entornos tediosos donde el rendimiento es crucial, o no se puede contar con un intérprete de Python. & \\ \hline

\textbf{C}++ & Orientado a objetos & Gran rendimiento, acceso a todas las características de C & No es portable fácilmente en algunos casos & Se han creado los bindings de MarcoPolo para este lenguaje, así como las herramientas \textbf{marcobootstrap}\\ \hline

\textbf{Java} & Orientado a objetos & Multiplataforma, popular y sencillo de utilizar & El rendimiento del lenguaje y su \textit{JVM} en el sistema son inferiores al de otras alternativas & Se utiliza en los paquetes \textit{software} que hacen uso de \textbf{Tomcat} o \textbf{Hadoop}.\\ \hline
\textbf{Perl} & Multiplataforma y multiparadigma & Portable, sencillo de utilizar, diseñado para la administración de sistemas & El uso de Perl como lenguaje de programación principal puede dificultar la realización de una serie de tareas clave. & Ninguna\\ \hline
\end{tabular}
\caption{Lenguajes de programación utilizados en el sistema}
\end{table}
\end{landscape}

\paragraph{Otros lenguajes\\}

Todas las interfaces web han sido programadas utilizando HTML, CSS y JavaScript en el lado del cliente. Dicha combinación evita la dependencia con cualquier herramienta no incluida por defecto en la totalidad de los navegadores mayoritarios (tales como Flash, ActiveX\dots).

Para la realización de diversas tareas de administración del sistema se utilizan \textit{scripts} de \textbf{bash}. 

\section{Herramientas utilizadas para la creación de \textit{software}}

\subsection{Twisted}

\textbf{Twisted} \citationneeded{https://twistedmatrix.com/} es un motor dirigido por eventos para la creación de aplicaciones basadas en red. Uno de los principales beneficios de la programación orientada a eventos es la capacidad del sistema de optimizar el tiempo de CPU y evitar cambios de contexto, pues todo el código se ejecuta en un único hilo. Twisted se basa en el patrón de diseño \textbf{reactor}\cite{Coplien95reactor}, que se basa en la gestión de diferentes eventos, su demultiplexación y el envío a los manejadores apropiados de forma síncrona (ver \ref{eventdriven}).

Twisted permite crear de forma sencilla sockets asíncronos a bajo nivel en los protocolos UDP y TCP y aplicaciones que utilizan protocolos bien definidos, como HTTP o DNS. Es capaz de trabajar con protocolos como \textbf{multicast} o \textbf{TLS} e integra funcionalidades para el desarrollo dirigido por pruebas (\textit{test-driven development}).

\textbf{Twisted} se ha utilizado para la creación de la herramienta de descubrimiento de servicios \textbf{MarcoPolo} (ver \ref{marcopolo}).

\subsection{Tornado}

\textbf{Tornado} es un \textit{framework} web y una biblioteca para aplicaciones en red que utiliza mecanismos de entrada/salida asíncrona, permitiendo crear herramientas como \textbf{WebSockets} de forma sencilla y escalable. Todo el código, a menos que explícitamente se indique lo contrario se ejecuta en un único hilo.

\textbf{Tornado} se utiliza en todas las interfaces web creadas, en ocasiones en conjunción con \textbf{Django} y se integra con \textbf{MarcoPolo} a través del \textit{binding} para Python.

\subsection{Websockets}

El protocolo WebSocket \cite{rfc6455} posibilita el establecimiento de un canal bidireccional en una arquitectura cliente-servidor sobre el protocolo HTTP/HTTPS evitando el uso de peticiones asíncronas (XmlHttpRequest, <iframe>) y \textit{polling}.

La mayoría de las interfaces web creadas utilizan este tipo de comunicación para obtener información desde los diferentes nodos del sistema, optimizando la comunicación al reducirse el intercambio de datos al momento en el que estos son necesarios (al contrario de otras estrategias) y posibilitando la difusión de eventos en directo, a diferencia de estrategias como el \textit{polling}.

\subsection{PAM, LDAP}

Siguiendo el objetivo funcional\citationneeded, la gestión de los usuarios está delegada al sistema de autenticación preexistente en la infraestructura en la que se integra el sistema. En ella se cuenta con un servidor \textbf{LDAP} con la información de los usuarios. Mediante la configuración del paquete LDAP es posible accedeer a los mismos, y gracias a \textbf{PAM} (\textit{Pluggable Authentication Module}) se integra con el resto de métodos de autenticación presentes en el sistema.

Se ha creado un módulo para \textbf{PAM} para facilitar las tareas que el sistema debe realizar. Dicho módulo integra \textbf{MarcoPolo} y es conocido como \textbf{pam\_mkpolohomedir}\ref{pam_mkpolohomedir}. 

\subsection{OpenSSL}

OpenSSL es la implementación de código abierto más popular de los protocolos SSL y TSL. En el sistema se utiliza de forma intensiva para garantizar la confidencialidad de las transmisiones entre partes así como para verificar la identidad en ambos lados de un canal de comunicación. La biblioteca proporciona bindings a C, C++, Java y Python, por lo que su integración en cualquiera de las herramientas creadas ha sido trivial. 

\subsection{Distcc}

Distcc es la herramienta utilizada para el desarrollo del compilador distribuido. %TODO

\subsection{Hadoop}
%TODO

\subsection{\textit{Message Passing Interface}}

La necesidad de una herramienta de comunicación independiente de la plataforma derivó en la especificación del estándar MPI\cite{MPISpec}, un conjunto de interfaces para la creación de aplicaciones paralelas mediante la gestión de las operaciones de entrada-salida, definición de tipos de datos, grupos de proceso, creación y gestión de procesos, interfaces externas, etcétera. La especificación se define independientemente del lenguaje, si bien incluye implementaciones en C, C++, Fortran 77 y Fortran 95.

MPI se ha convertido con el paso de los años en la interfaz de referencia para la creación de aplicaciones distribuidas, contando con varias implementaciones como \textbf{MPICH} (la implementación original) u \textbf{OpenMPI} (presente en la mayoría de supercomputadores), de tipo libre, o implementaciones propietarias, tales como IBM MPI, Intel MPI, Cray MPI, Microsoft MPI, Myricom MPI.

MPI es utilizado en el sistema como herramienta de desarrollo de aplicaciones distribuidas, utilizando \textbf{MarcoPolo} para simplificar el proceso de descubrimiento de nodos (ver \ref{marcodiscover}). Además se han creado herramientas accesorias para facilitar varias tareas generalmente necesarias durante el desarrollo con la biblioteca (ver \ref{marcoinstallkey}).

\subsection{Tomcat}

Las prácticas de la asignatura Sistemas Distribuidos se realizan sobre el contenedor de servicios Tomcat. Por ello el sistema ofrece instancias de Tomcat a todos los usuarios sin que estos deban realizar ningún tipo de configuración. Además, se habilita el uso de Tomcat en herramientas como \textbf{deployer} para facilitar el desarrollo de las prácticas.

\section{Herramientas utilizadas para la gestión de código, calidad de \textit{software} y el proyecto}

\subsection{Git}

Git es un sistema de control de versiones capaz de gestionar proyectos de cualquier escala, diseñado para flujos de trabajo distribuidos. Git permite realizar operaciones de reversión de cambios, bifurcaciones y uniones de flujos de desarrollo y gestión de varias copias independientes sin conflictos. Es una de las herramientas de gestión de código más utilizadas, siendo diseñada originalmente para la coordinación en el desarrollo del núcleo Linux.

Las diferentes herramientas que componen el sistema son creadas en repositorios independientes de código. Dicho código es almacenado en un servidor con el objetivo de facilitar la movilidad del código entre las diferentes máquinas que componen el sistema y a modo de copia de seguridad. Se sigue un modelo de desarrollo basado en ramas que representan características a añadir a la versión estable, revisión de código, documentación o mantenimiento y solución a \textit{bugs}.

Todos los repositorios de código originales pueden consultarse en \href{https://bitbucket.org/Alternhuman}{https://bitbucket.org/Alternhuman}

\subsection{Redmine}

Redmine es una herramienta de gestión de proyectos basada en web que permite a un equipo mantener un registro de todo el trabajo realizado y planificado en un proyecto, con una serie de herramientas como diagramas de Gantt, Wiki o integración con sistemas de control de versiones.

El proyecto cuenta con una instancia de Redmine alojada en \href{http://redmine.martinarroyo.net/}{http://redmine.martinarroyo.net/projects/tfg}. En dicha instancia se han registrado todos los avances en el desarrollo del proyecto desde las fases iniciales del mismo.


\subsection{2to3, 3to2}

Las versiones del lenguaje Python 2 y 3 son incompatibles entre sí. Sin embargo, las diferencias entre ambas versiones radican en una serie de variaciones sintácticas, nombres de tipos y localización de las bibliotecas básicas del sistema, por lo que escribiendo código que tenga en cuenta dichas variaciones, bien incluyendo sentencias condicionales en función de la versión o bien utilizando mecanismos que sean compatibles con ambas versiones es posible generar código compatible con ambas versiones. Este aspecto de las herramientas es importante si se tiene en cuenta que la transición a la versión 3 aún no está completa.

2to3 y 3to2 son herramientas que ayudan al programador en la verificación de la compatibilidad entre versiones, generando una lista de modificaciones que posibilitan la ejecución el código en otra versión. Utilizando ambas herramientas es posible crear código ambivalente de forma sencilla.

\subsection{Pylint}

Pylint es una herramienta de verificación de código, que evalúa la calidad de un código siguiendo una serie de criterios tales como la presencia de errores sintácticos, sangrado del código, convenciones de nombrado y de estilo, errores en la importación de paquetes, etcétera. Pylint incluye además la herramienta pyreverse, útil en la generación de diagramas UML.

\subsection{Desarrollo dirigido por pruebas: Unittest, CppUnit, Trial}

Uno de los mecanismos para la detección temprana de errores es el desarrollo dirigido por pruebas (ver \ref{teoria:tdd}). En el proyecto se utilizan diferentes \textit{frameworks} para cada una de las aplicaciones y bibliotecas creadas. Los tests unitarios se incluyen en cada uno de los paquetes para poder ser ejecutados por cualquier usuario en caso de que lo estime oportuno.

\section{Herramientas de modelado}

Junto a Pylint se ha utilizado Visual Paradigm para la generación de todos los diagramas UML.


\section{Herramientas utilizadas para la documentación del proyecto}

\subsection{{\LaTeX}}

El sistema de composición de textos {\LaTeX} es utilizado para crear todos los documentos incluidos en el desarrollo del sistema. Se utiliza el motor {\XeLaTeX} para el compilado de los ficheros, debido a su mayor variedad de fuentes y la utilización por defecto de la codificación UTF-8 (útil en idiomas que utilizan un alfabeto diferente al del inglés). 

Se utiliza {\BibTeX} como gestor de la bibliografía en todos los documentos producidos.

\subsection{Sphinx}

Sphinx es un sistema de creación y generación de documentación capaz de crear documentos en diferentes formatos (HTML, \LaTeX, ePub\dots) a partir de una serie de archivos en el formato reStructuredText. Es además capaz de crear documentación sobre código Python (lenguaje para el que la herramienta fue creada) a partir de los comentarios presentes en el código (conocidos como \textbf{docstrings}). Junto con Doxygen, se ha conseguido documentar código en C, C++ y %TODO Java
utilizando Sphinx. Soporta además referencias a diferentes proyectos creados con esta herramienta.

Los resultados generados por la herramienta son incluidos como documentación técnica de cada una de las herramientas, y están disponibles en \href{marcopolo.martinarroyo.net}.


%Esta parte de la memoria tiene como objetivo presentar las técnicas metodológicas y las herramientas de desarrollo que se han utilizado para llevar a cabo el proyecto. Si se han estudiado diferentes alternativas de metodologías, herramientas, bibliotecas se puede hacer un resumen de los aspectos más destacados de cada alternativa, incluyendo comparativas entre las distintas opciones y una justificación de las elecciones realizadas. No se pretende que este apartado se convierta en un capítulo de un libro dedicado a cada una de las alternativas, sino comentar los aspectos más destacados de cada opción, con un repaso somero a los fundamentos esenciales y referencias bibliográficas para que el lector pueda ampliar su conocimiento sobre el tema

\subsection{Doxygen}

Doxygen es un sistema similar a Sphinx utilizado en proyectos escritos en C, C++ y Java entre otros muchos lenguajes. Constituye el estándar \textit{de facto} para la generación de documentación.

En el proyecto Doxygen es utilizado para documentar aquellas partes del proyecto que Sphinx no puede procesar (actualmente el soporte de dicha herramienta se limita a Python). Posteriormente la documentación de ambas herramientas se combina mediante ficheros XML generados por Doxygen que Sphinx puede procesar.

\section{Herramientas para la gestión de usuarios}

\subsection{SSH-HPN}

Una de las características de OpenSSH es que todas las tareas se ejecutan en un único proceso y por tanto, en un único núcleo, constituyendo un cuello de botella que se hace notable en sistemas de bajas prestaciones, como el sistema a modelar.

Con el objetivo de superar este límite nace SSH-HPN\citationneeded, un conjunto de modificaciones al código fuente de OpenSSH que optimiza la ejecución del mismo mediante el uso de diferentes procesos repartidos en los diferentes núcleos del sistema. El proyecto se distribuye como un archivo \texttt{.diff} que se incluye en los archivos del código fuente con la herramienta \textbf{GNU patch}.

Se ha creado un paquete de Arch Linux con el código fuente ya preparado para trabajar en la arquitectura ARM, y todos los nodos del sistema utilizan esta versión en lugar del paquete \textbf{OpenSSH} original.

\section{Otras herramientas}

%TODOArchivo BibTeX con todas las referencias de las RFCs: \href{http://tm.uka.de/~bless/bibrfcindex.html}{http://tm.uka.de/~bless/bibrfcindex.html}
