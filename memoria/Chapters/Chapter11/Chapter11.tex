\chapter{Evaluación de usuarios}

Uno de los procesos escogidos para la constatación de la efectividad de las diferentes propuestas de solución a problemas planteados por los alumnos, profesores y administradores es la realización de evaluaciones de los prototipos creados así como las versiones finales de los productos \textit{software} y \textit{hardware}.

Dichos procesos de evaluación han sido muy efectivos a la hora de detectar potenciales fallos en las interfaces de usuario que empobrecen de forma significativa la experiencia de estos con el sistema. Además, permiten conocer necesidades no identificadas que se han traducido en nuevas características del sistema.

Se han seguido diferentes estrategias de evaluación. El método más utilizado en la fase de captura de requisitos ha sido la realización de entrevistas. Posteriormente se han utilizado prototipos de las herramientas en las fases de evaluación, tanto de forma guiada (se muestran los pasos al usuario y se evalúa su opinión) como autónoma (se observa el comportamiento del usuario con el sistema).

Tras cada evaluación se realiza un análisis, incluyendo en un informe todos los aspectos de relevancia extraídos de la sesión. Las conclusiones de dicho informe determinan las acciones a llevar a cabo.

Se ha realizado un total de \citationneeded[Número de evaluaciones]. El informe de cada una de ellas puede consultarse en el anexo correspondiente (ver \ref{anexo:evaluaciones}).