\lhead{\emph{Introducción}}
\chapter{Introducción}

La presente memoria recoge el proceso de instalación de un sistema distribuido formado por dispositivos Raspberry Pi y la creación de un conjunto de protocolos, herramientas y programas para la utilización del mismo como herramienta de investigación en el campo de la computación distribuida y como herramienta didáctica para disciplinas relacionadas con dicho área.

El sistema se compone de un conjunto de dispositivos físicos compuesto por los nodos de computación y una serie de módulos accesorios, así como los diferentes mecanismos de alimentación y refrigeración, un conjunto de herramientas software que permiten la coordinación y comunicación entre los diferentes procesos y una serie de herramientas que facilitan el trabajo con el sistema.

\section{Contenidos de la memoria}

\begin{itemize}

\item Definición del dominio del problema y motivación

\item Evaluación de alternativas y propuesta de solución

\item Plataforma física

\item Herramientas creadas

	\subitem MarcoPolo
	\subitem MarcoTools
	\subitem MarcoStatusMonitor
	\subitem Deployer

	\subitem Material didáctico
		\subsubitem Ricard Agrawala


\item Aplicaciones distribuidas

	\subitem MPI
	\subitem Python
	\subitem Tomcat

\item Evaluación
	\item Consulta a los estudiantes
	\item Evaluación de las prácticas en MPI
	\item Evaluación de las prácticas en Sistemas Distribuidos
\end{itemize}