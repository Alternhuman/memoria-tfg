\lhead{\emph{Servicios auxiliares}}
\chapter{Servicios auxiliares}


\section{Compilador cruzado}

Si bien el sistema Raspberry Pi es capaz de compilar el \textit{software} que después utilizará, en ocasiones es beneficioso delegar dicha tarea a otro componente que realice el proceso por el nodo en cuestión y posteriormente añadir los archivos ejecutables al sistema. Este enfoque reduce el tiempo de trabajo de forma significativa, como observaremos posteriormente.%proporcione los resultados al solicitante.

\subsection{Creación de un compilador cruzado}

Un compilador cruzado (\textit{cross-compiler})es una herramienta capaz de generar código para una architectura utilizando un equipo con otra arquitectura diferente.%TODO: http://en.wikipedia.org/wiki/Cross_compiler
El uso de compiladores cruzados 

\subsection{Análisis del rendimiento}

Para determinar el rendimiento del compilador se ha utilizado el mismo en la compilación de diferentes herramientas a utilizar en el sistema:

\subsubsection{OpenMPI}

\textbf{Tiempo de compilación en Rasbperry Pi}: 4447 seconds (1 h y 14 minutos)

\textbf{Tiempo de compilación con el compilador cruzado sin paralelización}: 2710 seconds (45 minutos)

\textbf{Tiempo de compilación con 4 trabajos paralelos (make -j4)}: 1267 seconds (21 minutos)

\subsubsection{OpenCV}

\subsubsection{OpenSSH}

\section{Servidor marcobootstrap}